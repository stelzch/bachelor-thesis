%% LaTeX2e class for student theses
%% sections/conclusion.tex
%% 
%% Karlsruhe Institute of Technology
%% Institute for Program Structures and Data Organization
%% Chair for Software Design and Quality (SDQ)
%%
%% Dr.-Ing. Erik Burger
%% burger@kit.edu
%%
%% Version 1.3.5, 2020-06-26

\chapter{Conclusion}
\label{ch:Conclusion}

Binary Tree Reduction offers reproducible results independent of the core-count.
It is not limited to floating-point operations and can be extended to the general set of reduction operations.
Its message count per \gls{pe} is bounded logarithmically by the number of elements per \gls{pe} (\Cref{eq:upperBoundLowerDistribution}).
If the Binary Tree Reduction takes up the majority of the runtime, optimizing the data distribution for the reduction can yield better results (\Cref{sec:DataDistribution}).

For floating-point summation in particular, solutions like ReproBLAS~\cite{ahrens_algorithms_2020} outperform Binary Tree Reduction and have a negligible runtime penalty compared to naive \texttt{MPI\_Allreduce} implementations.
The slowdown of Binary Tree Reduction compared to ReproBLAS is typically less than $2$.

Future work could explore additional optimizations of the Binary Tree Reduction algorithm.
Under the assumption of a specific data distribution, $\rankFromIndex$-lookups could be replaced with a constant time algorithm.
Furthermore, critical path analysis could provide better-performing calculation orders for outbound subtree roots.
While \Cref{ch:Experiments} provides insight in the runtime of isolated reduction operations, additional examinations must be made to determine the runtimes of reductions which are a small part of larger workloads.