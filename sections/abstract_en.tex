%% LaTeX2e class for student theses
%% sections/abstract_en.tex
%% 
%% Karlsruhe Institute of Technology
%% Institute for Program Structures and Data Organization
%% Chair for Software Design and Quality (SDQ)
%%
%% Dr.-Ing. Erik Burger
%% burger@kit.edu
%%
%% Version 1.3.5, 2020-06-26

\Abstract
Because of rounding errors, parallel floating-point summation can produce different results on different core-counts.
For some algorithms like hill climbing, RAxML-NG~\cite{kozlov_raxml-ng_2019} or greedy algorithms, this implies that results may be irreproducible with different core-counts.
We present the Binary Tree Reduction algorithm, which follows a distributed binary tree scheme that keeps the calculation order fixed and independent of the core-count $p$.
A naive implementation requires up to $(p - 1) * (\log_2 \big(\frac{N-1}{p} \big) + 1)$ messages to sum $N$ floating-point numbers.
To reduce the message count, we introduce a message buffer and optimize data distribution across the cores, the latter results in a runtime decrease of $18\,\%$.
We find that for $p=256$, Binary Tree Reduction has a slowdown of less than $2$ compared to a naive, irreproducible solution.
It is able to compute the sum of $N=21 * 10^6$ summands on $p=256$ cores in about \SI{248}{\micro\second}.