\chapter{Introduction}
\label{ch:Introduction}

\section{Motivation}
\label{sec:Motivation}
A common problem in massively parallel computations is the reduction (for example summation) of results over the entire cluster.
Widespread implementations do not account for cluster-size-independent reproducibility and will deliver different results
even if the only variable element is the number of participating \glsfirstplural{pe}.
Irreproducibility of reduction operations propagates upwards into the results of high-level scientific software packages, impeding researchers abilities to understand and exchange these results.

RAxML-NG~\cite{kozlov_raxml-ng_2019} for example is a software package that searches for the most likely phylogenetic trees based on biological input sequences.
Given the exponentially large number of possible trees (for $100$ taxa there exist over $10^{182}$ distinct phylogenies~\cite{stamatakis_efficient_2020}) and proven $\mathcal{NP}$-hardness of the problem~\cite{roch_short_2006}, a exhaustive tree search is infeasible.
Instead, RAxML-NG uses stochastic evolutionary models to determine the likelihood of a given tree and performs a tree search to find a maximum likelihood estimate.
Because of their small magnitude, programs usually deal with likelihood values logarithmically.
To increase execution speed, RAxML-NG assumes that different sites evolve independently and computes per-site likelihoods in parallel on different threads.
The overall likelihood of a tree is the product of all per-site likelihoods, therefore the tree log-likelihood is the sum of all \glspl{psllh}:
\begin{align}
L_{\textrm{tree}} &= \prod_{s \in \textrm{sites}} L_s \\
\label{eq:llh_sum}
\log L_{\textrm{tree}} &= \log \left(\prod_{s \in \textrm{sites}} L_s\right) = \sum_{s \in \textrm{sites}}  \log L_s
\end{align}
The search path is chosen based on the log-likelihood of the current candidate tree.
Therefore, the correctness of sum \eqref{eq:llh_sum} is critically important for the resulting trees.
RAxML-NG uses IEEE 754 floating-point numbers~\cite{noauthor_ieee_nodate} to represent \glspl{psllh} values, but floating-point arithmetic is not necessarily associative due to rounding errors~\cite{goldberg_what_1991}.

Darriba et al.~\cite{darriba_state_2018} have shown that because of different summation orders, executing the same version of RAxML-NG with the same input data and same random seed can still produce different trees if the number of threads varies.
Diethelm~\cite{diethelm_limits_2012} reports on the irreproducibility of a software used to simulate sheet metal forming and identifies two common causes: sums whose order is determined by the time in which \glspl{pe} finish intermediate results and different propagation of rounding errors for a varied number of \glspl{pe}.
Wiesenberger et al.~\cite{wiesenberger_reproducibility_2019} study the irreproducibility of the \textsc{FELTOR} software package used for fluid simulations due to floating-point non-associativity and counter the problem by deriving bitwise-reproducible subroutines.