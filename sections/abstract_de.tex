%% LaTeX2e class for student theses
%% sections/abstract_de.tex
%% 
%% Karlsruhe Institute of Technology
%% Institute for Program Structures and Data Organization
%% Chair for Software Design and Quality (SDQ)
%%
%% Dr.-Ing. Erik Burger
%% burger@kit.edu
%%
%% Version 1.3.5, 2020-06-26

\Abstract
Die Addition von Gleitkommazahlen kann aufgrund von Rundungsfehlern bei unterschiedlicher Prozessorenanzahl zu unterschiedlichen Ergebnissen führen.
Für manche Algorithmen wie RAxML-NG~\cite{kozlov_raxml-ng_2019} oder Greedy-Algorithmen kann dies den Verlust der Reproduzierbarkeit bei unterschiedlicher Prozessorenanzahl bedeuten.
Wir stellen einen Reduktionsalgorithmus vor der nach dem Schema eines verteilten Binärbaums vorgeht, wodurch die Ausführungsreihenfolge unabhängig von der Prozessorenanzahl $p$ bleibt.
Eine naive Implementierung muss bis zu $(p - 1) * (\log_2 (\tfrac{N-1}{p}) + 1)$ Nachrichten senden um $N$ Gleitkommazahlen zu addieren.
Um die Nachrichtenanzahl zu senken führen wir einen Nachrichtenpuffer ein und optimieren die Datenverteilung über die Prozessoren, wobei letzteres zur einer Verringerung der Laufzeit um $18\,\%$ führt.
Wir stellen fest, dass für $p=256$ Prozessoren der Binärbaum-Reduktionsalgorithmus weniger als $2$-mal langsamer ist als ein naiver, unreproduzierbarer Algorithmus.
Er ist imstande $N=21*10^6$ Summanden auf $p=256$ in \SI{248}{\micro\second} aufzusummieren.